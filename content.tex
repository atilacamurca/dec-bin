\begin{frame}\frametitle{}

\slidetitle{Introdução}

\end{frame}

\section{Introdução}

\begin{frame}\frametitle{Introdução}

O sistema de numeração decimal possui 10 dígitos daí a nomenclatura
decimal.

Entretanto na informática este sistema de numeração provou-se falho.
Então foi usado o sistema de numeração binário, o qual possui apenas os
valores 0 e 1.

\end{frame}

\begin{frame}\frametitle{}

\slidetitle{Representação Numérica}

\end{frame}

\section{Representação Numérica}

\begin{frame}\frametitle{Representação Numérica}

Dado um número $n$ na base $\beta$, isto é,
$n = (a_j \, a_{j-1} \dots a_2 \, a_1 \, a_0)_\beta$, sendo
$0 \le a_k \le \beta-1$ com $k = 0, \dots, j$ podemos representá-lo na
forma polinomial:

\begin{equation}
n = a_j \beta^j + a_{j-1} \beta^{j-1} + \dots + a_2 \beta^2 + a_1 \beta^1 + a_0 \beta^0
\end{equation}

\end{frame}

\subsection{Representação decimal}

\begin{frame}\frametitle{Representação decimal}

Assim para escrevermos o número $(815)_{10}$, fazemos $\beta = 10$:

\begin{equation}
8 \cdot 10^2 + 1 \cdot 10^1 + 5 \cdot 10^0 = 815
\end{equation}

\end{frame}

\subsection{Representação binária}

\begin{frame}\frametitle{Representação binária}

E quanto qual seria o número $(0011\, 0010\, 1111)_2$?

\begin{equation}
1 \cdot 2^9 + 1 \cdot 2^8 + 1 \cdot 2^5 + 1 \cdot 2^3 + 1 \cdot 2^2 + 1 \cdot 2^1 + 1 \cdot 2^0 = 815
\end{equation}

Temos o mesmo número representado de outra maneira.

\end{frame}

\begin{frame}\frametitle{}

\slidetitle{Conversão binário para decimal}

\end{frame}

\section{Conversão binário para decimal}

\begin{frame}\frametitle{Conversão binário para decimal}

É possível notar que de posse de uma representação numérica binária
podemos fazer a conversão para decimal usando a fórmula diretamente,
pois a representação decimal é a convenção utilizada pelo ser humano.
Mas e o contrário?

\end{frame}

\begin{frame}\frametitle{}

\slidetitle{Conversão decimal para binário}

\end{frame}

\section{Conversão decimal para binário}

\begin{frame}\frametitle{Conversão decimal para binário}

Considere o número $(42)_{10}$ e $(a_j \, a_{j-1} \dots a_1 \, a_0)_2$
sua representação binária. Pelo processo inverso temos:

\begin{eqnarray}
n_0 = 42 = 2 \cdot 21 + 0 = 2 \cdot n_0 + a_0 \Rightarrow a_0 &=& 0 \\
n_1 = 21 = 2 \cdot 10 + 1 = 2 \cdot n_1 + a_1 \Rightarrow a_1 &=& 1 \\
n_2 = 10 = 2 \cdot 5 + 0 = 2 \cdot n_2 + a_2 \Rightarrow a_2 &=& 0 \\
n_3 = 5 = 2 \cdot 2 + 1 = 2 \cdot n_3 + a_3 \Rightarrow a_3 &=& 1 \\
n_4 = 2 = 2 \cdot 1 + 0 = 2 \cdot n_4 + a_4 \Rightarrow a_4 &=& 0 \\
n_5 = 1 = 2 \cdot 0 + 1 = 2 \cdot n_5 + a_5 \Rightarrow a_5 &=& 1
\end{eqnarray}

Temos o número $(0010\, 1010)_2$.

\end{frame}

\begin{frame}\frametitle{Conversão decimal para binário}

Podemos a partir desse algorítimo, criar ferramentas que facilitam a
conversão. Por exemplo criar uma tabela do tipo:

\ctable[pos = H, center, botcap]{cccccccc}
{% notes
}
{% rows
\FL
128 & 64 & 32 & 16 & 8 & 4 & 2 & 1
\ML
0 & 0 & 1 & 0 & 1 & 0 & 1 & 0
\LL
}

Dessa forma seremos capazes de decompor o valor somando as potências de
2 até que o resultado seja o número esperado.

Nesse caso $32 + 8 + 2 = 42$

\end{frame}

\begin{frame}\frametitle{Referências}

\begin{itemize}
\item
  \href{http://www.mat.uel.br/plnatti/Calculo\%20Numerico/Aulas/Aula1-C\%C3\%A1lculo\%20Num\%C3\%A9rico\_Erros.ppt}{www.mat.uel.br
  - Erros - Representações na base decimal e binária}
\item
  \href{http://eltiger.wordpress.com/2011/10/08/macete-dos-alunos-conversao-binaria-para-decimal-e-vice-versa-sem-o-uso-de-muitos-calculos-matematicos/}{eltiger.wordpress.com
  - Conversão binária para decimal e vice-versa}
\item
  \href{http://fatosmatematicos.blogspot.com.br/2011/02/conversao-de-numeros-decimais-para-base.html}{fatosmatematicos.blogspot.com.br
  - Conversão de Números Decimais Para a Base Binária e Vice-Versa}
\item
  \href{http://www.infomaroto.com/blog/converter-decimais-em-binarios-usando-javascript/}{www.infomaroto.com
  - Converter decimais em binários usando Javascript}
\end{itemize}
\end{frame}
